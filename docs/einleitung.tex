\section{Einleitung}
Das Projekt ist nach meinem vorletzten Jugend-Forscht Projekt entstanden: Ich
habe ein Praktikum in astronomischen Rechen Institut in Heidelberg genutzt, um
mit einem Doktoranden\footnote{Tim Tugendhat} das Navarro-Frenk-White Profil,
das zum Generieren von Punkt Wolken genutzt wird, zu visualisieren.
Anschließend hat sich das Projekt ein bisschen verlaufen, irgendwann beschloss
ich jedoch, dass das Projekt weiterzuführen und statt nur statische Galaxien zu
generieren, dazu überzugehen die Galaxien zu simulieren, also die Entwicklung
einer virtuellen Galaxie zu untersuchen.\\\par Eines der entscheidenden
Probleme war die Laufzeit der Simulation. Das Problem, das es zu lösen galt,
war die ursprüngliche Laufzeit der Simulation von \(O(n^2)\) soweit zu
minimieren, sodass die Simulation einer ''echten'' Galaxie in absehbarer Zeit
durchführbar ist.\\\par Die Simulation von einem Zeitschritt bei 200 Millionen
Sternen würde es erfordern \(4 \cdot 10^{16} \) Kräfteberechnungen
durchzuführen. Im fall von 1.000.000 Berechnungen pro Sekunde wäre die
Berechnung für einen Zeitschritt nach ca. \textbf{1267 Jahren} fertig. Durch
viele Optimierungen schafft es meine Software die Anzahl an Kräften, die
berechnet werden müssen auf (bestenfalls) \( 2.7 \cdot 10^{9} \) zu reduzieren
und somit eine Laufzeit von ca. \textbf{45 minuten} zu erreichen.
