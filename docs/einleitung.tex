\section{Einleitung}
Das Projekt ist nach meinem vorletzten Jugend-Forscht Projekt entstanden: Ich
habe ein Praktikum in Astronomischen Rechen Institut in Heidelberg genutzt, um
mit einem Doktoranden\footnote{Tim Tugendhat} das Navarro-Frenk-White Profil,
das zum generieren von Punkt Wolken genutzt wird, zu visualisieren. Anschließend
hat sich das Projekt ein bisschen verlaufen, irgendwann beschloss ich jedoch,
dass das Projekt weiterzuführen und statt nur statische Galaxien zu generieren
dazu überzugehen die Galaxien zu simulieren, also die Entwicklung einer
virtuellen Galaxie zu untersuchen.  Eines der Entscheidenden Probleme war die
Laufzeit der Simulation. Das Problem das es zu lösen galt, war die Nutzung von
mehreren Threads mit der Nutzung des Barnes-Hut Algorithmus zu kombinieren.
Das Ergebnis ist sehr schön: Durch die Nutzung der Programmiersprache Go
war das einbauen der Nutzung von mehreren Threads vergleichsweise einfach.

