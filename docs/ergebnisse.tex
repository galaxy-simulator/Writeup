\section{Ergebnisse}

Die Generierung der Punktwolken ist komplett skaliert, es ist nun möglich
mehrere Generator-Instanzen hochzufen welche die Sterne generieren und in eine
Datenbank schreiben. Die Sterne in der Datenbank können nun auch simuliert
werden, die ``ursprüngliche`` Laufzeit in \( O(n^2) \) ist auf \( O(n \cdot
log_4(n)) \) reduziert was es (in der Theorie) ermöglicht eine ``echte``
Galaxie mit 200.000.000 Sternen in \textbf{45 Minuten} statt \textbf{1267
Jahren} zu simulieren (Faktor 14.808.695). Es wird dabei davon ausgegangen,
dass pro Sekunde die Kraft die auf 1.000.000 Sterne wirkt berechnet werden
kann. Dies ist auf einen einzlnem Rechner nicht durchführbar, durch die
Aufteilung auf mehrere Rechner ist es jedoch möglich.
