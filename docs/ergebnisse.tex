\section{Ergebnisse}

Die Generierung der Punktwolken ist komplett skaliert, es ist nun möglich
mehrere Generator-Instanzen hochzufahren welche die Sterne generieren und in
eine Datenbank schreiben. Die Sterne in der Datenbank können auch simuliert
werden, die ``ursprüngliche`` Laufzeit in \( O(n^2) \) ist auf \( O(n \cdot
log_4(n)) \) reduziert was es (in der Theorie) ermöglicht eine ``echte``
Galaxie mit 200.000.000 Sternen in \textbf{45 Minuten} statt \textbf{1267
Jahren} zu simulieren (Faktor 14.808.695). Es wird dabei davon ausgegangen,
dass pro Sekunde die Kraft die auf 1.000.000 Sterne wirkt berechnet werden
kann. Dies ist auf einem einzlnem Rechner (derzeitig) nicht durchführbar, durch
die Aufteilung auf mehrere Rechner ist dies jedoch möglich.

\par Durch die Aufteitung in mehrere Module welche containerisiert sind, kann
die Simulation mithilfe von Crouwdsourcing (in der theorie) durch mehrere
freiwillige helfer einfach genutzt werden, wodurch es möglich wird viele
personen in das projekt mit einzubeziehen, jedoch nicht zu überlasten.
