\section{Containerisierung und  Modularisierung}

Um eine optimale Skalierbarkeit zu erreichen wird die Anwendung in einzelne
Module aufgeteilt und in einzelne Container verpackt. Dadurch ist es einfach
möglich die Anwendung auf mehreren Rechnern gleichzeitig laufen zu lassen und
entsprechende Interaktionen zwischen den Container zu definieren.

\subsection{Modularisierung des Generators}
Um den Generator zu modularisieren muss erst definiert werden was für
potentielle Module existieren und wie diese miteinander interagieren.

\par Insgesamt generiert der Generator zufällige Werte in einem gegebenen
Intervall, testet mithilfe des NFW-profils ob diese Sterne existieren oder
nicht und schreibt die Sterne anschließend in eine Datenbank. Es sind sofort
ein paar Module ersichtlich: ein Modul welches die Zufälligen Koordinaten
generiert, ein Modul welches den Wert aus dem NFW-Profil berechnet und ein
Modul welches die Daten in die Datenbank schreibt.

\begin{figure}[ht!]
\centering
\begin{forest}
    for tree={draw, grow=0}
    [DB
        [generator
            [traefik
                [NFW]
                [\( \dots \)]
                [NFW]
            ]
        ]
    ]
\end{forest}
\label{fig:generator_setup}
\end{figure}

\subsubsection{Generator Modul}
Das Generator Modul generiert zufällige Koordinaten in einem definiertem
Intervall und sendet diese an einen NFW Container.  Damit nicht ein Container
unter der last der ein kommenden Antworten leidet wird der reverse-Proxy
Traefik\footnote{\url{https://traefik.io/}} verwendet. Dieser routet die
Anfragen an weitere Container weiter wodurch optimale Lastverteilung und
Skalierbarkeit gegeben ist.

\subsubsection{NFW Modul}
Das NFW modul erhält einen Wert und berechnet den entsprechenden NFW Wert.
Dadurch das er durch Traefik angesteuert wird kann falls die Anzahl der
Anfragen zu hoch wird einfach ein identische Container gestartet werden.
Traefik erkennt diesen Container automatisch und kann diesen beim routen der
Anfragen entsprechend nutzen.

\subsubsection{DB Modul (1)}

\subsection{Modularisierung des Simulators}
Der Simulator simuliert die Sterne aus der Datenbank indem er Stern für Stern
die Kraft die auf einen Stern wirkt berechnet, die neue Position des Sternes
ausrechnet und anschließend den ``neuen'' Stern zurück in die Datenbank
schriebt.

\begin{figure}[ht!]
\centering
\begin{forest}
    for tree={draw, grow=0}
    [DB
        [DB-actions
            [manager
                [Simulator]
                [\( \dots \)]
                [Simulator]
            ]
        ]
    ]
\end{forest}
\label{fig:simulator_setup}
\end{figure}

\subsubsection{Manager}
Um die Simulations Container optimal skalieren zu können, wird statt den
Simulations Container aktiv Sterne zu geben darauf gewartet, dass ein
Simulations Container einen Stern anfragt. Der Manager fragt im Vorhinein die
Datenbank an um eine Liste an Stern-IDs zu bekommen auf die die Kraft berechnet
werden müssen. Diese Liste an Stern-IDs wird in einen Channel geschrieben,
welcher die Sterne einzeln ausgeben kann.  Sobald der Channel leer ist,
entnimmt der Manager der Datenbank die nächsten Stern-IDs.

\subsubsection{Simulator}
Der Simulator Container entnimmt dem Manager Container einen Stern und
berechnet die Kraft die auf ihn wirkt indem er den in der Datenbank
gespeicherten Baum in Kombination des Barnes-Hut Algorithmus nutzt. Nachdem die
Kraft berechnet wurde kann die Neue Position des Stenes berechnet werden und
wieder in die Datenbank eingefügt werden.

\subsubsection{DB Modul (2)}
Der Datenbank Container interagiert mit der Datenbank und stellt verschiedene
Methoden zur Verfügung um z.B. Sterne in die Datenbank einzufügen, Daten aus
der Datenbank zu erhalten und den Massen Mittelpunkt aller inneren Knoten zu
berechnen.

\subsection{Sonstige Container}

\subsubsection{Viewer}
\subsubsection{Controller}

\subsection{Datenbank Skalierung}

Ein Bottleneck das bei der Skallierung enersteht ist die Anbinding and die Datenbank: Desto mehr Simulations-Container mit der Datenbank interagieren, desto höher wird die Auslastung der Datenbank. 

\subsubsection{Sharding}
Problem: viele simulations container wollen mit einer Datenbank reden

Lösung: Datenbank sharding

\subsubsection{Caching}
Problem: Bandbreite zu niedrig

Lösung: Local Caching (Geoabhängig)
