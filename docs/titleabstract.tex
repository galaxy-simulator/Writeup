% title
\title{\huge Galaxy Simulation\\ \large Jugend Forscht 2018/2019}
\date{2018 - 2019}
\author{\Large{Emile Hansmaennel}\\ Theodor-Fliedner-Gymnasium, Heinrich Heine
Universität Düsseldorf}

% title with an abstract in a single column
\twocolumn[
    \maketitle
    \centering
    \begin{minipage}{0.7\textwidth}
        Ist es möglich die Entstehung von Galaxien zu simulieren? Um diese
        Frage zu beantworten bin ich zu dem Schluss gekommen, dass ich das doch mal
        ausprobieren sollte. Dazu habe ich das Navarro-Frenk-White Profil implementiert
        um Cluster an Sternen zu generieren und anschließen die Kräfte die Zwischen den
        Sternen wirken zu berechnen. Dabei stattete ich die Sterne mit einer zufälligen
        Masse aus und Unterteilte die Galaxie in dynamisch-große Zellen um die
        Simulation stark zu beschleunigen. Um die Simulation noch stärker zu optimieren,
        implementierte ich die Simulation sehr modular um diese auf theoretisch
        mehreren Tausend Servern gleichzeitig laufen zu lassen. Insgesamt sollte es nun
        möglich sein einen Zeitschritt in einer Galaxie mit 200 Millionen Sternen in
        ca. \textit{45 Minuten} statt \textit{1265 Jahren} zu berechnen (Angenommen es
        werden 1 Millionen Kraftberechnungen pro Sekunde durchgeführt).
    \end{minipage}
    \bigskip
    \bigskip
]



