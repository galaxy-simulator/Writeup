% title
\title{\huge Galaxy Simulation\\ \large Jugend Forscht 2018/2019}
\date{2018 - 2019}
\author{\Large{Emile Hansmaennel}\\ Theodor-Fliedner-Gymnasium, Heinrich Heine
Universität Düsseldorf}

% title with an abstract in a single column
\twocolumn[
    \maketitle
    \centering
    \begin{minipage}{0.7\textwidth}
Ist es möglich die Entstehung von Galaxien zu simulieren?  Um diese Frage zu
beantworten bin ich zu dem Schluss gekommen, dass ich das doch einfach mal
ausprobieren sollte. Dazu habe ich das Navarro-Frenk-White Profil implementiert
um anschließen die Kräfte die Zwischen den Sternen wirken zu berechnen. Dabei
stattete ich die Sterne mit einer zufälligen Masse aus und Unterteilte die
Galaxie in dynamisch-große Zellen um die Simulation stark zu beschleunigen.  Um
eine Stabile Galaxie zu simulieren müssen jedoch alle Sterne in der Galaxie
eine Anfangsgeschwindigkeit besitzen die sie auf eine Kreisbahn lenkt, damit
die Kraft, welche die Sterne in die Mitte der Galaxie zieht ausgeglichen
werden.
    \end{minipage}
    \bigskip
    \bigskip
]



